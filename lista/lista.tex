\documentclass[a4paper]{article}

\usepackage[T1]{fontenc}
\usepackage{textcomp}
\usepackage[italian]{babel}
\usepackage{amsmath, amssymb, amsthm}
% for \lightning
\usepackage{stmaryrd}
\usepackage{geometry}
\geometry{
left=10mm,
right=10mm,
}
\usepackage{tikz-cd}
\usepackage{multicol}

\newcommand{\R}{\mathbb{R}}
\newcommand{\C}{\mathbb{C}}
\newcommand{\Q}{\mathbb{Q}}
\newcommand{\N}{\mathbb{N}}
\newcommand{\A}{\mathbb{A}}
\newcommand{\Z}{\mathbb{Z}}

\renewcommand{\Im}{\operatorname{Im}}

\newcommand\numberthis{\addtocounter{equation}{1}\tag{\theequation}}

\newtheorem{theorem}{Theorem}
\newtheorem{lemma}{Lemma}
\newtheorem{definition}{Definition}

\newcounter{examplecounter}
\newenvironment{example}{%
\refstepcounter{examplecounter}%
\textbf{Example \arabic{examplecounter}}%
}

\begin{document}
\title{Elenco di risultati di ETI}
\author{Francesco Minnocci}
\date{\vspace{-4ex}}
\maketitle
\begin{multicols}{2}
\begin{itemize}
	\item Assiomi
	\item Proposizione: \(\nexists\) l'insieme di tutti gli insiemi
	\item (Buona) definizione: Coppia di Kuratowski
	\item Proposizione: Esistenza del prodotto cartesiano
	\item Definizione: Relazione d'ordine
	\item Definizione: Assiomi di Peano
	\item Teorema: \((\omega,\emptyset,\text{succ})\) soddisfa gli Assiomi di Peano
	\item Teorema: \(<\;\equiv\;\in\) è un ordine totale su \(\omega\)
	\item Teorema: Induzione forte
	\item Teorema: Principio del minimo
	\item Teorema: Ricorsione numerabile
	\item Teorema: ogni modello di Peano è isomorfo ad \((\omega,\emptyset,\text{succ})\)
	\item Teorema: ricorsione numerabile v2
	\item Teorema: Cantor-Bernstein
	\item Teorema: \(|A|<|\mathcal{P}\left( A \right) |\)
	\item Definizione: Operazioni sulle cardinalità
	\item Proposizione: finito \(\implies\) Dedekind-finito
	\item Proposizione: \(A\) infinito \(\implies\forall n\in\omega,~n<|A|\)
	\item Proposizione \(|A|\leq|\omega|\implies A\) numerabile o finito
	\item Lemma: \(\left( A,\;< \right) \) totale finito \(\implies A\sim n\)
	\item Lemma: Caratterizzazione dei buoni ordini isomorfi ad \(\omega\)
	\item Proposizione: \(\omega\times\omega\sim\omega\)
	\item Definizione: \(\Z, \Q\)
	\item Teorema: di isomorfismo di Cantor
	\item Definizione: \(\R\) tramite sezioni di Dedekind
	\item Proposizione: \(\R\) è totalmente ordinato e completo
	\item Teorema: \(|\R|=2^{\aleph_0}=|\mathcal{P}\left( \N \right)| \)
	\item Lemma: \(\left( A,\;< \right) \) completo senza estremi e un denso numerabile è isomorfo ad \(\R\)
	\item Teorema: Tricotomia della relazione d'ordine fra buoni ordini
	\item Definizione: Operazioni sui buoni ordini
	\item Definizione: Ordinali
	\item Teorema: Tricotomia degli ordinali
	\item Proposizione: \(s\left( x \right) \) è il minimo ordinale maggiore di \(x\)
	\item Proposizione: esistenza di minimo e sup di un insieme di ordinali
	\item Proposizione: gli ordinali sono una classe propria
	\item Definizione: funzione classe
	\item Teorema: ogni buon ordine è isomorfo ad un unico ordinale
	\item Induzione transfinita v1 e v2
	\item Ricorsione transfinita v1 e v2
	\item Definizione: operazioni fra ordinali
	\item Teorema: equivalenza tra operazioni fra ordinali e fra buoni ordini
	\item Proposizione: monotonia delle operazioni fra ordinali
	\item Proposizione: Esistenza della sottrazione fra ordinali
	\item Lemma: Esistenza della divisione euclidea fra ordinali
	\item Teorema: Forma normale di Cantor
	\item Definizione: Funzioni classi continue \(\text{Ord}\to\text{Ord}\)
	\item Esistenza di frequenti punti fissi di una funzione crescente continua
	\item Teorema: Hartogs
	\item Definizione: numero di Hartogs
	\item Proposizione: \(\forall X ~~ \text{H}(X)\) è un ordinale iniziale
	\item Proposizione: gli ordinali iniziali sono una classe propria
	\item Proposizione: ogni ordinale iniziale infinito è limite
	\item Proposizione: \(\forall C\) classe propria di ordinali \(\exists!\;F:\text{Ord}\to \text{C}\) crescente biettiva
	\item Definizione: \(\omega_\alpha\)
	\item Proposizione: banalità delle operazioni fra Aleph
	\item Proposizione: AC \(\iff\) ogni funzione suriettiva ha una sezione
	\item Teorema: equivalenza fra AC, Zorn e Zermelo
	\item Proposizione: ogni insieme è equipotente ad un ordinale
	\item Teorema: Tarski
	\item Corollario: finito \(\iff\) Dedekind-finito
	\item Proposizione: \(0<|A|\leq|B|\iff\exists\) \begin{tikzcd} A \arrow[r, twoheadrightarrow] & B \end{tikzcd}
	\item Proposizione: unione \(\aleph_\alpha\) di \(\aleph_\alpha\) è \(\aleph_\alpha\)
	\item Proposizione: Un ordine totale è u buono sse non ha successioni strettamente decrescenti infinite
	\item Teorema (Cantor-Bendixon): un chiuso di \(\R\) è si spezza in un perfetto e un numerabile chiuso disgiunti
	\item Lemma: ogni perfetto non vuoto ha cardinalità \(2^{\aleph_0}\)
	\item Teorema: König
  \item Definizione (ed equivalenza): cofinalità v1 e v2
  \item Teorema: Hausdorff
  \item Definizione: Gerarchia di Von Neumann
  \item Proposizione: gli assiomi relativizzati a \(V_*\) sono veri
  \item Teorema: \(V=V_*\)
\end{itemize}
\end{multicols}
\end{document}
