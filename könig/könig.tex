\documentclass[a4paper]{article}

\usepackage[T1]{fontenc}
\usepackage{textcomp}
\usepackage[italian]{babel}
\usepackage{amsmath, amssymb, amsthm}
% for \lightning
\usepackage{stmaryrd}
\newcommand{\R}{\mathbb{R}}
\newcommand{\C}{\mathbb{C}}
\newcommand{\Q}{\mathbb{Q}}
\newcommand{\N}{\mathbb{N}}
\newcommand{\A}{\mathbb{A}}
\newcommand{\Z}{\mathbb{Z}}

\renewcommand{\Im}{\operatorname{Im}}

\newcommand\numberthis{\addtocounter{equation}{1}\tag{\theequation}}

\newtheorem{theorem}{Theorem}
\newtheorem{lemma}{Lemma}
\newtheorem{definition}{Definition}

\newcounter{examplecounter}
\newenvironment{example}{%
\refstepcounter{examplecounter}%
\textbf{Example \arabic{examplecounter}}%
}

\begin{document}
\title{Il Teorema di König}
\author{Francesco Minnocci}
\maketitle
\begin{theorem}(König)
Siano \(\{\mathbf{k}_i\}_{i \in I}, \{\lambda_i\}_{i \in I}\) cardinali tali che \(\forall i \in I, \mathbf{k}_i < \lambda_i\). Allora, \[ \sum_{i \in I}{\mathbf{k}_i} <\prod_{i \in I}{\lambda_i} .\]
\end{theorem}
\begin{proof}
	Siano
\begin{align*}
	&A= \bigcup{\{\mathbf{k}_i\times\{i\}\mid i\in I\}}\\
	&B=\{f\colon I\rightarrow \sup_{i\in I}{\lambda_i}\mid \forall i\in I, f{(i)}\in \lambda_i\} .
\end{align*}
	Supponiamo, per assurdo, che esista una funzione suriettiva \[ f\colon A \longrightarrow B \]
	Infatti, per l'assioma della scelta, ciò è equivalente a supporre che
	\[ |A|\geq|B|>0 ,\]
	dove l'insieme di destra è non vuoto visto che i \(\lambda_i\) sono tutti non nulli.\\
Ora, se \(i \in I\) definiamo
\begin{align*}
	f_i\colon &\mathbf{k}_i\longrightarrow\lambda_i\\
	     &\alpha\longmapsto f{\left( \alpha,i \right) }_{(i)}\in\lambda_i,
\end{align*}
e visto che per ipotesi \(f_i\) non può essere suriettiva, si ha \[(\lambda_i\setminus\Im{f_i})\neq\emptyset.\]
Quindi, usando AC costruiamo
\begin{align*}
	g\colon&I\longrightarrow\sup_{i\in I}{\lambda_i}\\
	&i\longmapsto g{(i)}\in\left( \lambda_i\setminus\Im{f_i} \right)
.\end{align*}
Ora, \(g\in B\), ma mostriamo che \(g\notin\Im{(f_i)}\) (ovvero \(f\) non è suriettiva, da cui la contraddizione): se per assurdo
\begin{align*}
	\exists i\in I, \alpha\in\mathbf{k}_i\mid g=f{(\alpha,i)} \implies g{(i)}&=f{(\alpha,i)}_{(i)}\\
										 &=f_i{(\alpha)}\in\Im{f_i},
\end{align*}
che è una contraddizione, per costruzione di \(g\). \(\lightning\)
\end{proof}
\end{document}
